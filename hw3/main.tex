\documentclass[a4paper]{article}

%% Language and font encodings
\usepackage[english]{babel}
\usepackage[utf8x]{inputenc}
\usepackage[T1]{fontenc}

%% Sets page size and margins
\usepackage[a4paper,top=3cm,bottom=2cm,left=3cm,right=3cm,marginparwidth=2cm]{geometry}

%% Useful packages
\usepackage{algorithm}
\usepackage{amsfonts}
\usepackage{amsmath}
\usepackage{amssymb}
\usepackage{amsthm}
\usepackage{bbm}
\usepackage[colorinlistoftodos]{todonotes}
% \usepackage[colorlinks=true, allcolors=blue]{hyperref}
\usepackage{enumerate}
\usepackage{float}
\usepackage{graphicx}
\usepackage{mathrsfs}
\usepackage{subcaption}
\usepackage{tikz}
\usepackage{tikzscale}
\usetikzlibrary{shapes.geometric, arrows}
\tikzset{
    vertex/.style={circle,draw,minimum size=1.5em},
    edge/.style={->,> = latex'}
}
\tikzstyle{triger} = [circle, minimum width=2cm, minimum height=1cm, text centered, draw=black]
\tikzstyle{process} = [rectangle, minimum width=1cm, minimum height=1cm, text centered, draw=black]
\tikzstyle{decision} = [diamond, minimum width=2cm, minimum height=1cm, text centered, draw=black]
\tikzstyle{block} = [rectangle, minimum width=3cm, minimum height=3cm, text centered, draw=black]
\tikzstyle{arrow} = [thick,->,>=stealth]

\title{HW3}
\author{Kevin Chang}

\newtheorem{definition}{Definition}
\newtheorem{problem}{Problem}
\newtheorem{property}{Property}[section]
\newtheorem{theorem}{Theorem}[section]
\newtheorem{suspect}{Suspect}[section]
\newtheorem{example}{Example}
\newtheorem{lemma}[theorem]{Lemma}

\graphicspath{ {./images/} }

\begin{document}
\maketitle

\section{}
Recall that for a vector $w \in \mathbb{R}^d$ , $\mathcal{H}_w := \{z : <w, z> = 0\}$.
Let $S = \{(x_i , y_i )\}$ be a set of linearly separable data in $\mathbb{R}^d$ (i.e., $x_i \in \mathbb{R}^d$ and $y_i \in \{-1, 1\}$).
Define the set $\mathcal{M}_S$ to be the set of all vectors which separate the data with large dot product:
$$\mathcal{M}_S = \{w : y_i <w, x_i > \geq 1 \text{for} i = 1, . . . , n\}.$$
\begin{itemize}
    \item Let $w^*$ denote the element of $\mathcal{M}_S$ with smallest norm.
Show that for any other $w$ that separates the data $$\min \mathit{dist}_{1\leq i\leq n}(x_i , \mathcal{H}_w ) \leq \min_{1\leq i\leq n} \mathit{dist}(x_i , \mathcal{H}_{w^*}) .$$

Recall that for any nonzero vector \( w \in \mathbb{R}^d \), the distance from a point \( x \) to the hyperplane 
\(\mathcal{H}_w := \{z : \langle w, z \rangle = 0\}\) is given by
\[
\operatorname{dist}(x, \mathcal{H}_w) = \frac{|\langle w, x \rangle|}{\|w\|}.
\]
If \( w \) separates the data, then \( y_i \langle w, x_i \rangle > 0 \) for all \( i \), hence
\[
\min_i \operatorname{dist}(x_i, \mathcal{H}_w)
= \frac{\min_i y_i \langle w, x_i \rangle}{\|w\|}.
\]

Define
\[
\mathcal{M}_S = \{ w \in \mathbb{R}^d : y_i \langle w, x_i \rangle \ge 1, \; i = 1, \ldots, n \}.
\]
Let \( w^* \in \mathcal{M}_S \) be the element of smallest norm.  
For any separating \( w \), define
\[
\gamma := \min_i y_i \langle w, x_i \rangle > 0, 
\quad \text{and} \quad 
\tilde{w} := \frac{w}{\gamma}.
\]
Then \( \tilde{w} \in \mathcal{M}_S \), since
\[
y_i \langle \tilde{w}, x_i \rangle = \frac{y_i \langle w, x_i \rangle}{\gamma} \ge 1.
\]
Hence
\[
\min_i \operatorname{dist}(x_i, \mathcal{H}_w)
= \frac{\gamma}{\|w\|}
= \frac{1}{\|\tilde{w}\|}.
\]
Because \( w^* \) minimizes \( \|w\| \) over \( \mathcal{M}_S \),
\[
\|w^*\| \le \|\tilde{w}\|,
\]
which implies
\[
\min_i \operatorname{dist}(x_i, \mathcal{H}_w)
\le
\min_i \operatorname{dist}(x_i, \mathcal{H}_{w^*}).
\]
Thus, \( w^* \) achieves the \emph{maximum margin} among all separating hyperplanes.

    \item Show that there are real numbers $\alpha_i$ such that $w^* = \sum_{i=1}^n \alpha_i x_i$.


Consider the convex optimization problem:
\[
\min_{w} \frac{1}{2}\|w\|^2
\quad \text{subject to} \quad
y_i \langle w, x_i \rangle \ge 1, \; i = 1, \ldots, n.
\]
The Lagrangian is
\[
\mathcal{L}(w, \alpha) = \frac{1}{2}\|w\|^2 - \sum_{i=1}^n \alpha_i (y_i \langle w, x_i \rangle - 1),
\quad \alpha_i \ge 0.
\]
Setting the derivative with respect to \( w \) to zero (stationarity condition) yields:
\[
\nabla_w \mathcal{L} = w - \sum_{i=1}^n \alpha_i y_i x_i = 0
\quad \Rightarrow \quad
w^* = \sum_{i=1}^n \alpha_i y_i x_i.
\]
Thus, \( w^* \) lies in the span of the training examples \( \{x_i\} \).


    \item Show that the $\alpha_i$ can be chosen so that $y_i \alpha_i$ are all nonnegative.

Define \( \tilde{\alpha}_i := \alpha_i y_i \).
Then
\[
w^* = \sum_{i=1}^n \tilde{\alpha}_i x_i,
\quad \text{and} \quad
y_i \tilde{\alpha}_i = y_i^2 \alpha_i = \alpha_i \ge 0.
\]
Hence, the coefficients can be chosen so that \( y_i \alpha_i \ge 0 \) for all \( i \).

\end{itemize}

\section{}
Let $u$ and $v$ be $D$-dimensional unit vectors. Let $M$ be a random matrix of dimension $d \times D$.
Each entry of $M$ is generated iid from a normal distribution with mean 0 and variance $1/d$.
\begin{enumerate}
    \item Show that $\mathbb{E}[\langle M u, M v\rangle ] = \langle u, v\rangle$.
    \item Suppose $d \geq \frac{8}{\epsilon^2}$. Show that with probability at least $1 - e^{-1} - e^{-2}$,
        $$\langle M u, M v \rangle  \geq \langle u, v \rangle  - \epsilon.$$
    \item Now let’s apply this to machine learning. Consider a set of $n$ examples in $D$ dimensional space that is linearly separable with margin $y$.
        That is, there are $n$ examples, $(x_i , y_i )$ with $y_i \in \{-1, 1\}$ and $||x_i || \leq R$, and there is a unit vetor $w$ so that $y_i \langle w, x_i  \rangle  \geq y$ for all $i$.

        Suppose that $$d \geq 32 \frac{R^2}{\gamma^2}\log(4n).$$
        Show that with probability at least $1/2$, $y_i \langle M w, M x_i  \rangle  \geq \frac{\gamma}{2}$ for all $i$.
        We can think of the vectors $M x_i$ as embeddings of the original data set in a lower dimensional space.
        This problem shows a random embedding already preserves much of the linear separability of data.
        An optimized embedding can do only better.
    \item For parts 2 and 3, you can use the following fact about Gaussian random variables.
        If $g_1 , \dots, g_k$ are independent Gaussian random variables with mean zero and variance 1, then
        $$Pr\left[\frac{1}{m}\sum_{i=1}^m g_i^2 \geq 1 + \epsilon\right] \leq exp\left( -\frac{m\epsilon^2}{8}\right)$$
        $$Pr\left[\frac{1}{m}\sum_{i=1}^m g_i^2 \geq 1 - \epsilon\right] \leq exp\left( -\frac{m\epsilon^2}{4}\right)$$
\end{enumerate}

\section{}
Consider the function $k : (0, 1) \times (0, 1) \rightarrow \mathbb{R}$ defined by $k(x_1 , x_2 ) = min\{x_1 , x_2\}$.
\begin{enumerate}
    \item Prove that $k$ is a valid kernel (Hint: write $k$ as the integral of a product of two simple functions and then prove that its Gram matrices are positive semi-definite).
    \item Now, consider a training set $\{(x_i , y_i )\}_{i=1,\dots,n}$ with $y_i \in \mathbb{R}$ and distinct points $x_i$ in $(0, 1)$.
        Show that if we ran kernel regression without regularization on this data set, we would obtain zero training error.
More precisely, find explicit coefficients $\alpha_j$, in terms of the training data, such that for all points $(x_i , y_i)$ in the training set we have
$$\sum_{j=1}^n \alpha_j \min\{x_j , x_i \} = y_i .$$
\end{enumerate}


\end{document}

